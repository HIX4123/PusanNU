\documentclass[a4paper, titlepage]{article}
\usepackage[hangul]{kotex}
\usepackage{hyperref}
\usepackage{verbatim}

\hypersetup{
    colorlinks=true,
    linkcolor=blue,
    filecolor=magenta,      
    urlcolor=cyan,
}

\pagestyle{headings}
\title{인터넷과웹기초 프로젝트 최종보고서}
\author{202355517 권민규}
\date{\today}


\begin{document}
\maketitle

\section{개발 목표}
I'm feeling lucky

구글의 검색 창 밑에는 I'm feeling lucky라는 버튼이 있다. 이를 그냥 누르면 구글의 두들 사이트가 뜨고, 어떤 검색어를 입력한 상태에서 누르면 해당 검색 결과의 첫 번째 사이트로 즉시 이동한다.

이에서 착안해 I'm feeling lucky 기능을 네이버와 유튜브에 각각 구현해 본다.


\section{사용 API 소개}
Youtube Search API\\
\href{https://developers.google.com/youtube/v3/docs/search/list?hl=ko}{전자 열람 시 클릭으로 이동 가능}\\\\
Naver Search API\\
\href{https://developers.naver.com/apps/#/cooperation}{전자 열람 시 클릭으로 이동 가능}


\section{UX 시나리오 설정}

\subsection{UI 구성}

\paragraph{Form의 서로 다른 type의 input 3개 이상 사용}
\begin{enumerate}
    \item{text | }검색 키워드 입력
    \begin{verbatim}
index.html

<form id="form" onsubmit="func1()">
    <label for="key">Search keywords</label><br />
    <input type="text" id="key" name="key" autofocus />
</form>
    \end{verbatim}
    \item{select | }검색 대상 사이트 선택
    \begin{verbatim}
index.html

<form id="form" onsubmit="func1()">
    <select id="target" name="target">
        <option value="" selected>--Select--</option>
        <option value="naver">NAVER</option>
        <option value="youtube">YouTube</option>
    </select>
</form>
    \end{verbatim}
    \item{submit \& reset | }I'm feeling lucky, reset 버튼
    \begin{verbatim}
index.html

<form id="form" onsubmit="func1()">
    <input type="submit" value="I'm feeling lucky" />
    <input type="reset" value="Reset" />
</form>
    \end{verbatim}
\end{enumerate}


\paragraph{List 또는 Table 1개 이상 사용}

table | 검색 기록
\begin{verbatim}
index.html

<aside>
    <table border="1">
        <thead>
            <tr>
                <th>time</th>
                <th>keyword</th>
                <th>target</th>
            </tr>
        </thead>
        <tbody id="log"></tbody>
    </table>
</aside>
\end{verbatim}

\paragraph{Semantic Element ‘header’, ‘nav’, ‘footer’ 필수 사용}
코드 본문 참조

\paragraph{Attribute Selector 2개 이상}
\begin{verbatim}
styles.css


input[type="submit"] {
    background-color: orange;
}
input[type="submit"]:hover {
    background-color: red;
}

input[type="reset"] {
    background-color: greenyellow;
}
input[type="reset"]:hover {
    background-color: lime;
}
\end{verbatim}

\paragraph{Combinator 5개 이상}
\begin{verbatim}
styles.css

@media (max-width: 900px) {
    nav:hover ~ aside {
        display: block
    }
    nav:hover ~ main {
        display: none;
    }
}

youtubeEmbed.css

.embed-container iframe,
.embed-container object,
.embed-container embed {
    position: absolute;
    top: 0;
    left: 0;
    width: 70%;
    height: 70%;
    margin-top: 10px;
}
\end{verbatim}

\paragraph{Pseudo-class 2개 이상}
\begin{verbatim}
input[type="submit"]:hover {
    background-color: red;
}
input[type="reset"]:hover {
    background-color: lime;
}
nav:hover {
    color: aqua;
}
nav:hover ~ aside {
    display: block;
}
nav:hover ~ main {
    display: none;
}
nav:hover ~ .embed-container iframe {
    display: none;
}
\end{verbatim}

\paragraph{Pseudo-element 2개 이상}
\begin{verbatim}
styles.css

aside p::first-line {
    font-weight: bold;
    font-size: large;
}
\end{verbatim}

\paragraph{Float와 Clear 속성 사용 필수}
\begin{verbatim}
styles.css

aside {
    float: right;
    text-align: center;
    width: 30%;
}

main {
    clear: left;
    width: 100%;
    height: auto;
}
\end{verbatim}

\paragraph{모바일 환경 대응 (Media Query 사용, width: 900px 기준)}
\begin{verbatim}
styles.css

@media (max-width: 900px) {
    h1 {
        font-size: large;
    }
    nav {
        display: block;
    }
    aside {
        padding: 10px;
        float: none;
        width: 100%;
        display: none;
    }
    nav:hover {
        color: aqua;
    }
    nav:hover ~ aside {
        display: block;
    }
    form {
        width: 100%;
    }
    nav:hover ~ main {
        display: none;
    }
    footer {
        font-size: small;
    }
}

youtubeEmbed.css

@media (max-width: 900px) {
    .embed-container iframe {
        display: none;
    }
    .embed-container iframe,
    .embed-container object,
    .embed-container embed {
        position: absolute;
        top: 0;
        left: 0;
        width: 100%;
        height: 100%;
    }
    nav:hover ~ .embed-container iframe {
        display: none;
    }
}
\end{verbatim}

\subsection{UX 기능}
실행 방법\\
\verb|index.js| 실행 후 \href{http://localhost}{http://localhost} 접속
\begin{enumerate}
    \item 입력창에 검색어를 입력하고 NAVER를 선택해 I'm feeling lucky 버튼을 누르면 사이트 UI가 네이버의 이미지 컬러 계열로 바뀌고, 새 창에 네이버 검색결과에서 제일 먼저 나오는 사이트를 연다.
    \item 입력창에 검색어를 입력하고 YOUTUBE를 선택해 I'm feeling lucky 버튼을 누르면 사이트 UI가 유튜브의 이미지 컬러 계열로 바뀌고, 유튜브 검색결과에서 제일 먼저 나오는 사이트를 임베디드에 연다.
    \item 아무것도 입력하지 않은 상태에서 luck 버튼을 누르면 화면 아래쪽에 검색 기록을 띄운다. 사이트가 새로고침되면 기록은 사라진다.\\
          \textbf{$\Longrightarrow$ 검색 기록은 화면의 우측에 별도로 띄운다. 검색어나 타겟 사이트가 입력되지 않은 채 I'm feeling lucky 버튼을 누르면 경고창을 띄운다.}
    \item 모바일 검색 환경에서 사이트 접근 시 로그창이 사라지고 검색 폼의 우측 상단에 "Log" 텍스트가 뜬다. 마우스를 올리면 로그창이 나타난다.\\
          ※로그 텍스트의 판정 범위는 용이성을 위해 해당 텍스트를 포함한 가로줄 전체로 한다.
    \item reset 버튼을 누르면 입력창의 내용이 사라지고 UI 색상이 초기화된다.
\end{enumerate}

\subsection{추가 구현}
\begin{enumerate}
    \item 유튜브를 설정해 임베디드를 열 때 화면을 임베디드 위치로 스크롤한다.
          \begin{verbatim}
main.js

function youtube(key) {
    ...
    ...
    let youtubeEmbed = document.getElementById("YouTube video player");
    let xhr = new XMLHttpRequest();
    xhr.withCredentials = true;

    xhr.addEventListener("readystatechange", function () {
        if (this.readyState === 4) {
            ...
            youtubeEmbed.scrollIntoView();
        }
    });
    \end{verbatim}

    \item 포트 번호를 80번으로 설정해 URL에 포트 번호를 생략할 수 있게 하고, sendFile을 이용해 파일을 전송해 html에 접속할 때 경로를 입력하지 않아도 되게 한다.
          \begin{verbatim}
const express = require("express");
const app = express();
const port = 80;

app.use("/public", express.static("public"));

app.get("/", (_req, res) => {
    res.sendFile(__dirname + "/public/index.html");
});
app.get("/styles/styles.css", (_req, res) => {
    res.sendFile(__dirname + "/public/styles/styles.css");
});
app.get("/styles/youtubeEmbed.css", (_req, res) => {
    res.sendFile(__dirname + "/public/styles/youtubeEmbed.css");
});
app.get("/scripts/main.js", (_req, res) => {
    res.sendFile(__dirname + "/public/scripts/main.js");
});
    \end{verbatim}

    \item 한글 인코딩 오류를 해결하기 위해 encodeURI를 사용한다.
    \item 패비콘을 설정한다.
          \begin{verbatim}
<link
rel="apple-touch-icon"
sizes="57x57"
href="images/fabicon/apple-icon-57x57.png"
/>
...
...
...
    \end{verbatim}
\end{enumerate}
\end{document}