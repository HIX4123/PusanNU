\documentclass[a4paper, titlepage]{article}
\usepackage[hangul]{kotex}
\usepackage{hyperref}

\hypersetup{
    colorlinks=true,
    linkcolor=blue,
    filecolor=magenta,      
    urlcolor=cyan,
}

\pagestyle{headings}
\title{인터넷과웹기초 팀 프로젝트 중간보고서}
\author{202355517 권민규}
\date{\today}


\begin{document}
\maketitle

\section{개발 목표}
I'm feeling lucky

구글의 검색 창 밑에는 I'm feeling lucky라는 버튼이 있다. 이를 그냥 누르면 구글의 두들 사이트가 뜨고, 어떤 검색어를 입력한 상태에서 누르면 해당 검색 결과의 첫 번째 사이트로 즉시 이동한다.

이에서 착안해 I'm feeling lucky 기능을 네이버와 유튜브에 각각 구현해 본다.


\section{사용 API 소개}
Youtube Analytics API\\
\href{https://console.cloud.google.com/apis/library/youtubeanalytics.googleapis.com?project=oceanic-diagram-387216&supportedpurview=project}{전자 열람 시 클릭으로 이동 가능}\\\\
Naver Search API\\
\href{https://developers.naver.com/apps/#/cooperation}{전자 열람 시 클릭으로 이동 가능}


\section{UX 시나리오 설정}

\subsection{UI 구성}

\paragraph{Form의 서로 다른 type의 input 3개 이상 사용}
\begin{enumerate}
    \item{text | }검색 키워드 입력
    \item{select | }검색 대상 사이트 선택
    \item{button | }I'm feeling lucky 버튼
\end{enumerate}

\paragraph{List 또는 Table 1개 이상 사용}

table | 재생 목록

\subsection{UX 기능}
\begin{enumerate}
    \item site에서 Naver을 고른 채 input에 검색어를 입력하고 luck 버튼을 누르면 네이버 검색결과에서 제일 먼저 나오는 사이트를 새 창에 연다.
    \item site에서 Youtube을 고른 채 input에 검색어를 입력하고 luck 버튼을 누르면 유튜브 검색결과에서 제일 먼저 나오는 사이트를 임베디드에 연다.
    \item 아무것도 입력하지 않은 상태에서 luck 버튼을 누르면 화면 아래쪽에 검색 기록을 띄운다. 사이트가 새로고침되면 기록은 사라진다.
\end{enumerate}


\section{개발 계획 및 일정}

21주차 - HTML, CSS 등 프론트엔드 구성\\
22주자 - JavaScript, Node.js 등 백엔드 및 API구성\\
나머지 - 점검 및 보고서, 동영상 제작 후 제출

\end{document}